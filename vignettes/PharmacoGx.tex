%\VignetteIndexEntry{PharmacoGx: an R package for analysis of large pharmacogenomic datasets}
%\VignetteDepends{xtable}
%\VignetteSuggests{}
%\VignetteKeywords{}
%\VignettePackage{PharmacoGx}

\documentclass[11pt]{article}

\usepackage[utf8]{inputenc}
\usepackage{authblk}


\title{PharmacoGx: an R package for analysis of large pharmacogenomic datasets}
\author[1]{Petr Smirnov}
\author[1]{Zhaleh Safikhani}
\author[2]{Nehme El-Hechem}
\author[1]{Donald Wang}
\author[1]{Adrian She}
\author[1]{Catherina Olsen}
\author[1]{Deena Gendoo}
\author[3]{Patrick Grossman}
\author[4]{Andrew Beck}
\author[3]{Hugo Aerts}
\author[1]{Mathiew Lupien}
\author[5]{Anna Goldenburg}
\author[1]{Benjamin Haibe-Kains}
\affil[1]{Princess Margaret Cancer Centre, University Health Network, Toronto Canada}
\affil[2]{Institut de recherches clinque de Montr\'{e}al, Montr\'{e}al, Canada}
\affil[3]{Dana-Farber Cancer Institute, Boston, USA}
\affil[4]{Beth Israel Deaconess Medical Center, Boston, USA}
\affil[5]{Department of Computer Science, University of Toronto, Toronto, Canada}




\usepackage{Sweave}
\begin{document}
\Sconcordance{concordance:PharmacoGx.tex:/home/chris/rPackages/PharmacoGx/vignettes/PharmacoGx.Rnw:%
1 21 1 1 4 1 1 1 0 69 1 1 3 2 0 1 1 3 0 1 2 1 1 1 2 4 0 1 2 12 1 1 3 2 %
0 1 1 3 0 1 2 5 1 1 3 5 0 1 2 16 1 1 2 1 0 3 1 1 2 1 0 1 3 2 0 1 7 4 0 %
1 5 4 0 1 6 4 0 1 5 4 0 1 6 4 0 1 5 4 0 2 1 1 5 3 0 1 4 3 0 1 4 3 0 1 3 %
1 0 1 2 1 0 1 1 1 2 1 0 1 10 13 0 1 3 2 1 1 -5 1 12 13 1 1 2 1 0 2 1 1 %
3 2 0 1 1 1 7 5 0 1 2 2 1 1 2 20 0 1 2 45 1 1 3 1 0 1 3 2 0 1 8 7 0 1 8 %
7 0 4 1 16 0 1 2 51 1 1 2 31 0 1 2 4 1}


\maketitle
\tableofcontents

%------------------------------------------------------------
\section{Introduction}
%------------------------------------------------------------

Pharmacogenomics hold much potential to aid in discovering drug response
biomarkers and developing novel targeted therapies, leading to development of
precision medicine and working towards the goal of personalised therapy.
Several large experiments have been conducted, both to molecularly
characterise drug dose response across many cell lines, and to examine the
molecular response to drug administration. However, the experiments lack a
standardization of protocols and annotations, hindering meta-analysis across
several experiments.\\

\textit{PharmacoGx} was developed to address these challenges, by providing a
unified framework for downloading and analysing large pharmacogenomic datasets
which are extensively curated to ensure maximum overlap and consistency.
\textit{PharmacoGx} is based on a level of abstraction from the raw
experimental data, and allows bioinformaticians and biologists to work with
data at the level of genes, drugs and cell lines. This provides a more
intuitive interface and, in combination with unified curation, simplifies
analyses between multiple datasets.\\

To organize the data released by each experiment, we developed the
\textit{PharmacoSet} class. This class efficiently stores different types of
data and facilitates interogating the data by drug or cell line. The
\textit{PharmacoSet} is also versatile in its ability to deal with two
distinct types of pharmacogenomic datasets. The first type, known as
\textit{\'sensitivity\'} datasets, are datasets where cell lines were profiled
on the molecular level, and then tested for drug dose response. The second
type of dataset is the \textit{\'perturbation\'} dataset. These types of
datasets profile a cell line on the molecular level before and after
administration of a compound, to characterize the action of the compound on
the molecular level.\\

With the first release of \textit{PharmacoGx} we have fully curated and
created PharmacoSet objects for three publicly available large pharmacogenomic
datasets. Two of these datasets are of the \textit{sensitivity} type. These
are the Cancer Genome Project (CGP) \cite{Garnett:2012fc} and the Cancer Cell
Line Encyclopedia (CCLE) \cite{Barretina:2012fp}. The third dataset is of the
\textit{perturbation} type, the Connectivity Map (CMAP) project
\cite{Lamb:2006hf}.\\

Furthermore, PharmacoGx provides a suite of parallized functions which
facilitate drug response biomarker discovery, and molecular drug
characterization. This vignette will provide two example analysis case
studies. The first will be comparing gene expression and drug sensitivity
measures across the CCLE and CGP projects. The second case study will
interegate the CMAP database with a known signature of up and down regulated
genes for HDAC inhibitors as published in ~\cite{Glaser:2003gk}. Using the
Connectivity score as defined in ~\cite{Lamb:2006hf}, it will be seen that
known HDAC inhibitors have a high numerical score and high significance.\\ %%'check if the reference is right

For the purpose of this vignette, an extremely miniscule subset of all three
\textit{PharmacoSet} objects are included with the package as example data.
They are included for illustrative purposes only, and the results obtained
with them will likely be meaningless.\\


%------------------------------------------------------------
\subsection{Installation and Settings}
%------------------------------------------------------------

\textit{PharmacoGx} requires that several packages are installed. However, all dependencies are available from CRAN or Bioconductor, except for \textit{jetset}, which must be downloaded from the authors' website:

\begin{Schunk}
\begin{Sinput}
> install.packages(
+   'http://www.cbs.dtu.dk/biotools/jetset/current/jetset_3.0.0.tar.gz',
+    repos = NULL, type = 'source')
\end{Sinput}
\end{Schunk}

After installation of \textit{jetset}, the rest of the installation can be done by calling:

\begin{Schunk}
\begin{Sinput}
> source('http://bioconductor.org/biocLite.R')
> biocLite('PharmacoGx')
\end{Sinput}
\end{Schunk}

Load \textit{PharamacoGx} into your current workspace:
\begin{Schunk}
\begin{Sinput}
> library(PharmacoGx)
\end{Sinput}
\end{Schunk}

%------------------------------------------------------------
\subsection{Requirements}
%------------------------------------------------------------

\textit{PharmacoGx} has been tested on Windows and Mac platforms. However, it should compile on Linux. 

The packages utilisizes the core R package \textit{parallel} to preform parallel computations, and therefore if parallelization is desired, the dependencies for the parallel package must be met. 

%------------------------------------------------------------
\section{Downloading PharmacoSet objects}
%------------------------------------------------------------

We have made the PharmacoSet objects of the datasets he have curated available online at:

However, to make the process of obtaining the data easier through the R shell, we have also written a function \textit{getPSet} which automates downloading the datasets into a directory of the users choice, and returns the data into the R session. 

\begin{Schunk}
\begin{Sinput}
> x <- 'nothing'
\end{Sinput}
\end{Schunk}

%------------------------------------------------------------
\section{Sensitivity Case Study}
%------------------------------------------------------------

Our first case study illustrates the functions for analysis of the \textit{sensitivity} type of dataset. The case study will investigate the consistency between the CGP and CCLE datasets, recreating the analysis simlar to our paper \cite{HaibeKains:2013ie}. In both CCLE and CGP, the transcriptome of cells was profiled using an Affymatrix microarray chip. Cells were also tested for their response to increasing concetrations of various compounds, and form this the IC50 and AUC for cell mortality was computed. However, the cell and drugs names used between the two datasets were not consistent. Furthermore, two different microarray platforms were used. However, \textit{PharmacoGx} allows us to overcome these differences to do a comparative study between these two datasets. \\


CGP was profiled using the hgu133a platform, while CCLE was profiled with the expanded hgu133plus2 platform. While in this case the hgu133a is almost a strict subset of hgu133plus2 platform, we can use the best information from each platform by comparing the best probes within each dataset per gene. The function \textit{probeGeneMapping} helps select a probe for each gene profiled in the microarray platform. It provides two methods of probe selection, either by picking the most variant probe, which should be the probe most senstitive to variations in gene expression between cells, or by using jetset \cite{Li:2011cc}. Jetset scores each probe on how senesitive and specific it is, allowing probeGeneMapping to pick the probe which is most representative of each genes true expression. \\

To begin, you would load the datasets from disk or download them using the \textit{getPSet} function above, and then execute probeGeneMapping on the datasets. 
\begin{Schunk}
\begin{Sinput}
>   # If downloading the full datasets
> x <- 'nothing'
\end{Sinput}
\end{Schunk}

\begin{Schunk}
\begin{Sinput}
>   library(PharmacoGx)
>   data("CGPsmall")
>   data("CCLEsmall")
>   CGPsmall <- probeGeneMapping(CGPsmall, 
+                                platform='GPL96', 
+                                method='jetset')
>   CCLEsmall <- probeGeneMapping(CCLEsmall, 
+                                 platform = 'GPL97', 
+                                 method='jetset')
\end{Sinput}
\end{Schunk}
We then want to investigate the consistency of the data between the two datasets. The common intersection between the datasets can then be found using \textit{intersectPSet}. We then create a summary of the gene expression and drug sensitivity measures for both datasets, and compare them using a standard correlation coefficient. 

\begin{Schunk}
\begin{Sinput}
> library(PharmacoGx)
>   data("CGPsmall")
>   data("CCLEsmall")
>   CGPsmall <- probeGeneMapping(CGPsmall, 
+                                platform='GPL96', 
+                                method='jetset')
>   CCLEsmall <- probeGeneMapping(CCLEsmall, 
+                                 platform = 'GPL97', 
+                                 method='jetset')
>   common <- intersectPSet(list('CCLE'=CCLEsmall,
+                                'CGP'=CGPsmall),
+                           intersectOn=c("cell.lines", "drugs", 'genes'))
>   CGP.auc <- summarizeSensitivityPhenotype(
+                 common$CGP, 
+                 sensitivity.measure='auc_published', 
+                 summaryStat="median")
>   CCLE.auc <- summarizeSensitivityPhenotype(
+                 common$CCLE, 
+                 sensitivity.measure='auc_published', 
+                 summaryStat="median")
>   CGP.ic50 <- summarizeSensitivityPhenotype(
+                 common$CGP, 
+                 sensitivity.measure='ic50_published', 
+                 summaryStat="median")
>   CCLE.ic50 <- summarizeSensitivityPhenotype(
+                 common$CCLE, 
+                 sensitivity.measure='ic50_published', 
+                 summaryStat="median")
>   common$CGP <- summarizeGeneExpression(common$CGP, 
+                                         cellNames(common$CGP),
+                                         verbose=FALSE)
>   common$CCLE <- summarizeGeneExpression(common$CCLE, 
+                                          cellNames(common$CCLE),
+                                          verbose=FALSE)
>   gg <- geneNames(common[[1]])
>   cc <- cellNames(common[[1]])
>   ge.cor <- sapply(cc, function (x, d1, d2) {
+     return (cor(d1[ , x], d2[ , x], method="spearman",
+                 use="pairwise.complete.obs"))
+   }, d1=rnaData(common$CGP), d2=rnaData(common$CCLE))
>   ic50.cor <- sapply(cc, function (x, d1, d2) {
+     return (cor(d1[, x], d2[ , x], method="spearman",
+                 use="pairwise.complete.obs"))
+   }, d1=t(CGP.ic50), d2=t(CCLE.ic50))
>   auc.cor <- sapply(cc, function (x, d1, d2) {
+     return (cor(d1[ , x], d2[ , x], method="spearman",
+                 use="pairwise.complete.obs"))
+   }, d1=t(CGP.auc), d2=t(CCLE.auc))
>   w1 <- wilcox.test(x=ge.cor, y=auc.cor, conf.int=TRUE)
>   w2 <- wilcox.test(x=ge.cor, y=ic50.cor, conf.int=TRUE)
>   yylim <- c(-1, 1)
>   ss <- sprintf("GE vs. AUC = %.1E\nGE vs. IC50 = %.1E",
+                 w1$p.value, w2$p.value)
>   boxplot(list("GE"=ge.cor, "AUC"=auc.cor, "IC50"=ic50.cor),
+           main="Concordance between cell lines",
+           ylab=expression(R[s]),
+           sub=ss,
+           ylim=yylim,
+           col="lightgrey",
+           pch=20,
+           border="black")
> 
\end{Sinput}
\end{Schunk}

\begin{figure}[H!]
\begin{center}
